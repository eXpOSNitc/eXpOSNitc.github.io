\chapter{Synchronization and Access Control}
\label{chap5}

eXpOS assumes a single processor multi programming environment. This means that all processes exist concurrently in the machine and the OS time shares the machine between various processes. The OS specification requires that Round Robin scheduling is used with co-operative time-sharing. The OS does not provide any promise to the application program about the order in which processes will be executed.

However, application programs often need to stop and wait for another process to execute certain operations before proceeding. The OS provides system calls that allow user processes to synchronize execution.

\section{Process Synchronization}

eXpOS provides the Wait and Signal system calls for process synchronization. When a process executes the wait system call specifying the process id of another process as argument, the OS puts the calling process to sleep. This means, the OS will schedule the process out and won't execute it until one of the following events occur:
\begin{itemize}
\item The process specified as argument terminates by the exit system call.
\item The process specified as argument executes a Signal system call.

\end {itemize}

\section{Access Control}
A second major concurrency related requirement is that when multiple processes access the same data (in memory or files), it is often required to have some kind of locking mechanism to ensure that when one process is accessing the shared data, no other process is allowed to modify the same. This is to ensure data integrity and this issue is called the critical section problem.

eXpOS provides (binary) semaphores to allow application programs to handle the critical section problem. A (binary) semaphore is conceptually a binary-valued variable whose value can be set or reset only by the OS through designated system call. 

A process can acquire a semaphore using the Semget system call, which returns a unique semid for the semaphore. The semaphore is initially not set. Any of the processes sharing a semaphore identifier can set the semaphore (called locking) using the SemLock system call giving the semid as an argument. SemUnLock will reset (called release) the semaphore waking up all other processes which went to sleep trying to lock the semaphore. Just like semaphores, files can also be locked using the FLock and FUnLock system calls. 
\\ 
Semantics of Locking operation for files and semaphores:
\begin{itemize}

\item If the semaphore/file is already locked, the process goes to sleep and wakes up only when the semaphore/file is free.
\item Otherwise, process locks the semaphore/file and continues execution.

\end{itemize}
